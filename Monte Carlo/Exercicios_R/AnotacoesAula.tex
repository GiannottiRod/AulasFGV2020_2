% Options for packages loaded elsewhere
\PassOptionsToPackage{unicode}{hyperref}
\PassOptionsToPackage{hyphens}{url}
%
\documentclass[
]{article}
\usepackage{lmodern}
\usepackage{amssymb,amsmath}
\usepackage{ifxetex,ifluatex}
\ifnum 0\ifxetex 1\fi\ifluatex 1\fi=0 % if pdftex
  \usepackage[T1]{fontenc}
  \usepackage[utf8]{inputenc}
  \usepackage{textcomp} % provide euro and other symbols
\else % if luatex or xetex
  \usepackage{unicode-math}
  \defaultfontfeatures{Scale=MatchLowercase}
  \defaultfontfeatures[\rmfamily]{Ligatures=TeX,Scale=1}
\fi
% Use upquote if available, for straight quotes in verbatim environments
\IfFileExists{upquote.sty}{\usepackage{upquote}}{}
\IfFileExists{microtype.sty}{% use microtype if available
  \usepackage[]{microtype}
  \UseMicrotypeSet[protrusion]{basicmath} % disable protrusion for tt fonts
}{}
\makeatletter
\@ifundefined{KOMAClassName}{% if non-KOMA class
  \IfFileExists{parskip.sty}{%
    \usepackage{parskip}
  }{% else
    \setlength{\parindent}{0pt}
    \setlength{\parskip}{6pt plus 2pt minus 1pt}}
}{% if KOMA class
  \KOMAoptions{parskip=half}}
\makeatother
\usepackage{xcolor}
\IfFileExists{xurl.sty}{\usepackage{xurl}}{} % add URL line breaks if available
\IfFileExists{bookmark.sty}{\usepackage{bookmark}}{\usepackage{hyperref}}
\hypersetup{
  pdftitle={Anotacoes de aula},
  pdfauthor={Rodrigo Giannotti},
  hidelinks,
  pdfcreator={LaTeX via pandoc}}
\urlstyle{same} % disable monospaced font for URLs
\usepackage[margin=1in]{geometry}
\usepackage{color}
\usepackage{fancyvrb}
\newcommand{\VerbBar}{|}
\newcommand{\VERB}{\Verb[commandchars=\\\{\}]}
\DefineVerbatimEnvironment{Highlighting}{Verbatim}{commandchars=\\\{\}}
% Add ',fontsize=\small' for more characters per line
\usepackage{framed}
\definecolor{shadecolor}{RGB}{248,248,248}
\newenvironment{Shaded}{\begin{snugshade}}{\end{snugshade}}
\newcommand{\AlertTok}[1]{\textcolor[rgb]{0.94,0.16,0.16}{#1}}
\newcommand{\AnnotationTok}[1]{\textcolor[rgb]{0.56,0.35,0.01}{\textbf{\textit{#1}}}}
\newcommand{\AttributeTok}[1]{\textcolor[rgb]{0.77,0.63,0.00}{#1}}
\newcommand{\BaseNTok}[1]{\textcolor[rgb]{0.00,0.00,0.81}{#1}}
\newcommand{\BuiltInTok}[1]{#1}
\newcommand{\CharTok}[1]{\textcolor[rgb]{0.31,0.60,0.02}{#1}}
\newcommand{\CommentTok}[1]{\textcolor[rgb]{0.56,0.35,0.01}{\textit{#1}}}
\newcommand{\CommentVarTok}[1]{\textcolor[rgb]{0.56,0.35,0.01}{\textbf{\textit{#1}}}}
\newcommand{\ConstantTok}[1]{\textcolor[rgb]{0.00,0.00,0.00}{#1}}
\newcommand{\ControlFlowTok}[1]{\textcolor[rgb]{0.13,0.29,0.53}{\textbf{#1}}}
\newcommand{\DataTypeTok}[1]{\textcolor[rgb]{0.13,0.29,0.53}{#1}}
\newcommand{\DecValTok}[1]{\textcolor[rgb]{0.00,0.00,0.81}{#1}}
\newcommand{\DocumentationTok}[1]{\textcolor[rgb]{0.56,0.35,0.01}{\textbf{\textit{#1}}}}
\newcommand{\ErrorTok}[1]{\textcolor[rgb]{0.64,0.00,0.00}{\textbf{#1}}}
\newcommand{\ExtensionTok}[1]{#1}
\newcommand{\FloatTok}[1]{\textcolor[rgb]{0.00,0.00,0.81}{#1}}
\newcommand{\FunctionTok}[1]{\textcolor[rgb]{0.00,0.00,0.00}{#1}}
\newcommand{\ImportTok}[1]{#1}
\newcommand{\InformationTok}[1]{\textcolor[rgb]{0.56,0.35,0.01}{\textbf{\textit{#1}}}}
\newcommand{\KeywordTok}[1]{\textcolor[rgb]{0.13,0.29,0.53}{\textbf{#1}}}
\newcommand{\NormalTok}[1]{#1}
\newcommand{\OperatorTok}[1]{\textcolor[rgb]{0.81,0.36,0.00}{\textbf{#1}}}
\newcommand{\OtherTok}[1]{\textcolor[rgb]{0.56,0.35,0.01}{#1}}
\newcommand{\PreprocessorTok}[1]{\textcolor[rgb]{0.56,0.35,0.01}{\textit{#1}}}
\newcommand{\RegionMarkerTok}[1]{#1}
\newcommand{\SpecialCharTok}[1]{\textcolor[rgb]{0.00,0.00,0.00}{#1}}
\newcommand{\SpecialStringTok}[1]{\textcolor[rgb]{0.31,0.60,0.02}{#1}}
\newcommand{\StringTok}[1]{\textcolor[rgb]{0.31,0.60,0.02}{#1}}
\newcommand{\VariableTok}[1]{\textcolor[rgb]{0.00,0.00,0.00}{#1}}
\newcommand{\VerbatimStringTok}[1]{\textcolor[rgb]{0.31,0.60,0.02}{#1}}
\newcommand{\WarningTok}[1]{\textcolor[rgb]{0.56,0.35,0.01}{\textbf{\textit{#1}}}}
\usepackage{graphicx,grffile}
\makeatletter
\def\maxwidth{\ifdim\Gin@nat@width>\linewidth\linewidth\else\Gin@nat@width\fi}
\def\maxheight{\ifdim\Gin@nat@height>\textheight\textheight\else\Gin@nat@height\fi}
\makeatother
% Scale images if necessary, so that they will not overflow the page
% margins by default, and it is still possible to overwrite the defaults
% using explicit options in \includegraphics[width, height, ...]{}
\setkeys{Gin}{width=\maxwidth,height=\maxheight,keepaspectratio}
% Set default figure placement to htbp
\makeatletter
\def\fps@figure{htbp}
\makeatother
\setlength{\emergencystretch}{3em} % prevent overfull lines
\providecommand{\tightlist}{%
  \setlength{\itemsep}{0pt}\setlength{\parskip}{0pt}}
\setcounter{secnumdepth}{-\maxdimen} % remove section numbering

\title{Anotacoes de aula}
\author{Rodrigo Giannotti}
\date{}

\begin{document}
\maketitle

\hypertarget{aula-1009}{%
\section{Aula 10/09}\label{aula-1009}}

\hypertarget{metodos-de-aleatorizauxe7uxe3o}{%
\subsection{Metodos de
aleatorização}\label{metodos-de-aleatorizauxe7uxe3o}}

\begin{itemize}
\tightlist
\item
  \emph{stats}::\textbf{sample}: pega valores aleatoriamente de uma
  lista de valores disponíveis, com ou sem reposição
\item
  \emph{stats}::\textbf{runif} (r unif): gera valores uniformemente
  distribuidos
\item
  \emph{stats}::\textbf{rnorm}: gera valores normalmente distribuidos
\item
  \emph{stats}::\textbf{rbinom}: simula o numero de sucessos baseado no
  número de tentativas e a prob de sucesso
\item
  \emph{EnvStats}::\textbf{rtri}: gera valores triangularmente
  distribuidos
\item
  \emph{e1071}::\textbf{rdiscrete}: gera valores em uma distribuição
  discreta especificada
\end{itemize}

\emph{Vale notar que todas essas funções que começam com r possuem suas
variações começando por d, p e q, onde ao invés de número de observações
estas recebem arrays de quantis ou de probabilidades e possuem algumas
outras opções}

Exemplos:

\begin{Shaded}
\begin{Highlighting}[]
\CommentTok{# ?runif}
\NormalTok{unif <-}\StringTok{ }\KeywordTok{runif}\NormalTok{(}\DataTypeTok{n =} \DecValTok{10000}\NormalTok{, }\DataTypeTok{min =} \DecValTok{0}\NormalTok{, }\DataTypeTok{max =} \DecValTok{1}\NormalTok{)}
\NormalTok{unif }\OperatorTok\StringTok{ }\KeywordTok{hist}\NormalTok{(}\DataTypeTok{main =} \StringTok{"Uniforme"}\NormalTok{)}
\end{Highlighting}
\end{Shaded}

\includegraphics{AnotacoesAula_files/figure-latex/unnamed-chunk-1-1.pdf}

\begin{Shaded}
\begin{Highlighting}[]
\CommentTok{# ?rnorm}
\NormalTok{norm <-}\StringTok{ }\KeywordTok{rnorm}\NormalTok{(}\DataTypeTok{n =} \DecValTok{10000}\NormalTok{, }\DataTypeTok{mean =} \DecValTok{0}\NormalTok{, }\DataTypeTok{sd =} \DecValTok{1}\NormalTok{)}
\NormalTok{norm }\OperatorTok\StringTok{ }\KeywordTok{hist}\NormalTok{(}\DataTypeTok{breaks =} \DecValTok{50}\NormalTok{, }\DataTypeTok{main =} \StringTok{"Normal"}\NormalTok{)}
\end{Highlighting}
\end{Shaded}

\includegraphics{AnotacoesAula_files/figure-latex/unnamed-chunk-1-2.pdf}

\begin{Shaded}
\begin{Highlighting}[]
\CommentTok{# ?rbinom}
\NormalTok{binom <-}\StringTok{ }\KeywordTok{rbinom}\NormalTok{(}\DataTypeTok{n =} \DecValTok{10000}\NormalTok{, }\DataTypeTok{size =} \DecValTok{1}\NormalTok{, }\DataTypeTok{prob =} \FloatTok{.5}\NormalTok{)}
\NormalTok{binom }\OperatorTok\StringTok{ }\KeywordTok{hist}\NormalTok{(}\DataTypeTok{main =} \StringTok{"Binomial"}\NormalTok{)}
\end{Highlighting}
\end{Shaded}

\includegraphics{AnotacoesAula_files/figure-latex/unnamed-chunk-1-3.pdf}

\begin{Shaded}
\begin{Highlighting}[]
\CommentTok{# ?rtri}
\NormalTok{triang <-}\StringTok{ }\KeywordTok{rtri}\NormalTok{(}\DataTypeTok{n =} \DecValTok{10000}\NormalTok{, }\DataTypeTok{min =} \DecValTok{0}\NormalTok{, }\DataTypeTok{max =} \DecValTok{1}\NormalTok{, }\DataTypeTok{mode =} \FloatTok{0.8}\NormalTok{)}
\NormalTok{triang }\OperatorTok\StringTok{ }\KeywordTok{hist}\NormalTok{(}\DataTypeTok{breaks =} \DecValTok{50}\NormalTok{, }\DataTypeTok{main =} \StringTok{"Triangular"}\NormalTok{)}
\end{Highlighting}
\end{Shaded}

\includegraphics{AnotacoesAula_files/figure-latex/unnamed-chunk-1-4.pdf}

\begin{Shaded}
\begin{Highlighting}[]
\CommentTok{# ?rdiscrete}
\NormalTok{demanda <-}\StringTok{ }\DecValTok{14}\OperatorTok{:}\DecValTok{25}
\NormalTok{probabilidades <-}\StringTok{ }\KeywordTok{runif}\NormalTok{(}\KeywordTok{length}\NormalTok{(demanda))}
\NormalTok{discreta <-}\StringTok{ }\KeywordTok{rdiscrete}\NormalTok{(}\DataTypeTok{n =} \DecValTok{1000}\NormalTok{, }\DataTypeTok{probs =}\NormalTok{ probabilidades, }\DataTypeTok{values =}\NormalTok{ demanda)}
\KeywordTok{barplot}\NormalTok{(probabilidades, }\DataTypeTok{main =} \StringTok{"Probabilidades usadas na distribuição discreta"}\NormalTok{)}
\end{Highlighting}
\end{Shaded}

\includegraphics{AnotacoesAula_files/figure-latex/unnamed-chunk-1-5.pdf}

\begin{Shaded}
\begin{Highlighting}[]
\NormalTok{discreta }\OperatorTok\StringTok{ }\KeywordTok{hist}\NormalTok{(}\DataTypeTok{breaks =} \DecValTok{12}\NormalTok{, }\DataTypeTok{include.lowest =}\NormalTok{ T)}
\end{Highlighting}
\end{Shaded}

\includegraphics{AnotacoesAula_files/figure-latex/unnamed-chunk-1-6.pdf}

\end{document}
